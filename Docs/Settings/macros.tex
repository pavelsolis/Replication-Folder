%% Customized Macros
% Table of Contents, Tasks, Tables, Subcaptions, Text
%
%\gototoc
%\maketoc
%\begin{boxeditems} \end{boxeditems}
%\estauto
%\specialcell
%\fignotes
%\tabnotes
%\double
%\textchange

%---------------------------------------------------------------
% Table of Contents
%---------------------------------------------------------------

% Link to ToC from section
\newcommand{\gototoc}{\vspace{-2cm} \null\hfill [\hyperlink{toc}{Go to ToC}] \newline}

% Link back to section from ToC
\newcommand{\maketoc}{
	\clearpage
	\hypertarget{toc}{}
	\tableofcontents
	\thispagestyle{empty}
	\vspace{2.5\bigskipamount}
}

%---------------------------------------------------------------
% Tasks
%---------------------------------------------------------------

% Box with bullets for tasks to do in a section
\newenvironment{boxeditems}
	{\begin{tabular}{|p{\linewidth}|}
	\hline
	\begin{singlespace}
	\vspace{-0.4cm}
	\begin{itemize}
	}
	{
	\vspace{-0.4cm}
	\end{itemize}
	\end{singlespace}
	\\ \hline
	\end{tabular} \\
	}

%---------------------------------------------------------------
% Tables
%---------------------------------------------------------------

% Estout commands to handle fragments from Stata following Jörg Weber (https://www.jwe.cc/2012/03/stata-latex-tables-estout/)
\newcommand{\sym}[1]{\rlap{#1}}

\let\estinput=\input	% define new input command to flatten the document

\newcommand{\estauto}[2]{
	\newcolumntype{C}{>{\centering\arraybackslash}X}
	\vspace{.75ex}{
		\begin{tabularx}{0.95\linewidth}{l*{#2}C}
			\toprule
			\estinput{#1}
			\\ \bottomrule
			\addlinespace[.75ex]
		\end{tabularx}
	}
}

% Allow line breaks with \\ in table columns, e.g. mtitle("\specialcell{Co-Holding\\> \var1}")
\newcommand{\specialcell}[2][c]{\begin{tabular}[#1]{@{}c@{}}#2\end{tabular}}

%---------------------------------------------------------------
% Subcaptions
%---------------------------------------------------------------

% Notes after figures following Jörg Weber (https://www.jwe.cc/2012/03/stata-latex-tables-estout/)
\newcommand{\figtext}[1]{
	\vspace{-1ex}
	\captionsetup{justification=justified,font=footnotesize}
	\caption*{#1}
%	\captionsetup{justification=raggedright,singlelinecheck=false,font=footnotesize}
%	\caption*{\hspace{6pt}\hangindent=1.5em #1}
}

\newcommand{\fignote}[1]{\figtext{\emph{Note:~}~#1}}
\newcommand{\fignotes}[1]{\figtext{\emph{Notes:~}~#1}}

% Notes after tables
\newcommand{\tabnotes}[1]{
	\begin{tablenotes}[para,flushleft]
		\footnotesize \emph{Notes:~}~#1
	\end{tablenotes}
}

%---------------------------------------------------------------
% Text
%---------------------------------------------------------------

% Input numbers from file
\newcommand{\double}[1]{\input{#1}\unskip}

% Highlight changes in revised version with color
\newcommand{\textchange}[1]{\iftoggle{revised}{\textcolor{blue}{#1}}{#1}}