%% Packages for Paper
% Contents, Formatting, Math, Figures and Tables,
% Special Characters, Hyperlinks, References

%---------------------------------------------------------------
% Contents
%---------------------------------------------------------------
\usepackage[nottoc]{tocbibind} 		 			% nottoc omits ToC from itself
\usepackage{lipsum}						   		% To add dummy text
\usepackage{pdfpages}					 		% To insert (excerpts of) PDF files
\usepackage{appendix}							% options: titletoc, toc, title (b/f each letter), page (“Appendices” b/f first appendix)
%	\renewcommand*\appendixpagename{Appendix}
%	\renewcommand*\appendixtocname{Appendix}
%\usepackage{authblk}					 		% Multiple authors in blocks or in footnotes
%\usepackage{textcomp}		  			 		% Handles the copyright symbol
%\usepackage{makeidx} 							% Creates an index at the end, use \index{} and \printindex

%---------------------------------------------------------------
% Formatting
%---------------------------------------------------------------
\usepackage{pdflscape}			  		  		% To rotate page in PDF using \begin{landscape}
\usepackage{enumerate}				    		% To personalize the enumerate style
%\usepackage{datetime2}							% To customize date format

%---------------------------------------------------------------
% Math
%---------------------------------------------------------------
\usepackage{amsmath} 					 		% For multi-line statements
\usepackage{amssymb} 					 		% For math symbols and fonts
\usepackage{dsfont}								% More math fonts (e.g. indicator function)
%\usepackage{amsthm,enumitem}					% For theorems and control over list environments

%---------------------------------------------------------------
% Figures and Tables
%---------------------------------------------------------------
\usepackage{float}							 	% To enable H placement specifier (similar to h!)
\usepackage{standalone}			   				% To compile sub-files as part of main document
\usepackage{afterpage}					 		% To place floats after current page
\usepackage[labelsep=period,labelfont=bf]{caption} % Caption with dot separator, boldfaced
%	\captionsetup[figure]{position=top}			% To change position of caption
%	\captionsetup[table]{name=New Table Name}	% To change table caption prefix

% Figures
\usepackage{graphicx} 					 		% To include figures
\usepackage[outdir=./]{epstopdf}   				% To avoid errors when calling figures
\usepackage{subcaption}	    	  	   			% Multi-panel figure

% Tables (extensions to standard tabular environment)
\usepackage{tabularx}					 		% For paragraph-like columns
\usepackage{booktabs}			  	   			% For \toprule, \midrule, \bottomrule
\usepackage{multirow}			   				% To add entries with multiple rows
\usepackage{bigstrut} 			 	 		 	% To open up tabular spacing
\usepackage{siunitx}				 	  		% To align the decimal points
\usepackage{threeparttable}		 	  			% To include a structured note section
\usepackage{longtable} 			   				% For multi-page tables
%\usepackage{threeparttablex}		 			% Threeparttable functionality for long tables

%---------------------------------------------------------------
% Special Characters
%---------------------------------------------------------------
\usepackage[utf8]{inputenc}			  		 	% Handles accented characters (same output on all systems)
%\usepackage[T1]{fontenc}		  	  			% Fonts to use for printing characters
%\usepackage{underscore}				   		% Handles "_" in text but hurts filenames/labels with it

%---------------------------------------------------------------
% Hyperlinks
%---------------------------------------------------------------
\usepackage{xcolor}
	\definecolor{c1}{rgb}{0,0,0} 		   		% Black
	\definecolor{c2}{rgb}{0,0,1} 				% Blue
	\definecolor{c3}{rgb}{0,0,0.54} 		  	% Dark blue
\usepackage[colorlinks=true]{hyperref} 	 		% Creates hyperlinks, 'true' gets rid of the boxes
	\hypersetup{
		citecolor={c1}, 						% Citations
	    linkcolor={c2}, 						% Internal links
	    urlcolor={c3} 							% External links/URLs
	}

%---------------------------------------------------------------
% References
%---------------------------------------------------------------
% To read and process bibliographic data
\usepackage[style=authoryear,maxbibnames=99,maxcitenames=2,uniquelist=false,uniquename=false,dashed=false,backend=biber]{biblatex}
\usepackage[style=american]{csquotes}  			% For inline and display quotations

\addbibresource{../References/library.bib} 		%Imports the bibliography file 
\DeclareFieldFormat[book]{title}{#1}		  	% Non-italic book title
\DeclareFieldFormat[article]{pages}{#1} 		% No pp. before pages
\DeclareFieldFormat[incollection]{pages}{#1} 
\renewbibmacro{in:}{} 							% No in: before journal title
\xpretobibmacro{date+extradate}{\setunit{\addperiod\space}}{}{}	% Dot after author
\AtEveryBibitem{								% Omit fields
	\ifentrytype{article}{
		\clearfield{doi}%
		\clearfield{url}%
		\clearfield{urldate}%
		\clearfield{issn}%
		\clearfield{eprint}%
		\clearfield{review}%
		\clearfield{series}%%
		\clearfield{month}%%
	}{}
}
\renewbibmacro*{volume+number+eid}{%			% Redefine to remove dot after volume
	\printfield{volume}%
	\setunit*{\addnbspace}%
	\printfield{number}%
	\setunit{\addcomma\space}%
	\printfield{eid}}
\DeclareFieldFormat[article]{number}{\mkbibparens{#1}} % Parenthesis around volume number
\uspunctuation									% Activate ‘American-style’ punctuation
%\stdpunctuation								% Restores standard punctuation