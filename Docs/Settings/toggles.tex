%---------------------------------------------------------------
% Conditionals
%---------------------------------------------------------------
\usepackage{etoolbox}					 		% Toolbox of programming facilities
\usepackage{xpatch}						  		% Extends etoolbox patching commands

%---------------------------------------------------------------
% Paper
%---------------------------------------------------------------
\newtoggle{toclinks}				    	% When editing, add links (next to section titles) to ToC
\newtoggle{cboxes}					   		% When editing, write comments and to-dos
\newtoggle{fulldraft}					 	% Generate short (outline) and long (full draft) versions
\newtoggle{floatstxt}					 	% Put figures and tables in the text or at the end
\newtoggle{withappdx}					 	% Include appendix at the end of the paper
\newtoggle{blind}					 		% Generate version for journal submission (no identifiers)
\newtoggle{revised}					 		% Highlight changes in revised version

\settoggle{toclinks}{true}			   		% 'true' to include ToC and links
\settoggle{cboxes}{true}			   		% 'true' to include boxed comments
\settoggle{fulldraft}{true}	   	 	     	% 'false' to generate an outline
\settoggle{floatstxt}{true}	   	 	 		% 'true' to put figures and tables in the text
\settoggle{withappdx}{true}	   	 	 		% 'true' to include appendix at the end
\settoggle{blind}{false}	   	 	 		% 'true' to generate version without identifiers
\settoggle{revised}{false}	   	 	 		% 'true' to highlight changes with color

%---------------------------------------------------------------
% Slides
%---------------------------------------------------------------
\newtoggle{stops}					   		% Generate version with stepwise uncovering (more slides)

\settoggle{stops}{true} 					% 'false' for version without stops

%---------------------------------------------------------------
% Paper vs Slides
%---------------------------------------------------------------
\newtoggle{longnotes}					   	% Standalone descrptions in floats: Long for paper, short for slides
\newtoggle{coloreq}					   		% Highlight parts of an equation in slides

\settoggle{longnotes}{true} 				% No need to change it: paper.tex uses it as 'true', each float file sets it to 'false'
\settoggle{coloreq}{false} 					% No need to change it: paper.tex uses it as 'false', slides.tex sets it to 'true' when needed