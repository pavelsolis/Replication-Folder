\documentclass[12pt,a4paper]{article}
\usepackage[latin1]{inputenc}
\usepackage{amsmath}
\usepackage{amsfonts}
\usepackage{amssymb}
\usepackage{graphicx}
\usepackage{fancyhdr}
\usepackage{datetime2}
\usepackage[margin=1in]{geometry}
\setlength\parindent{0pt}

\fancyhf{}
\fancyhead[R]{Last Updated: \today}
\pagestyle{fancy}
%\renewcommand{\headrulewidth}{0pt} % remove awful top bar

\begin{document}
	\begin{center}
		\textbf{\Large Data Appendix for `Title of the Paper
% It is called by paper.tex, slides.tex, datastats.tex and tasks.tex
'}\\
		\medskip
		Author
	\end{center}

The Data Appendix serves as a codebook and users' guide for \textit{your analysis} data files. It provides information about \textbf{all} the variables in the analysis data files, such as their names, definitions, coding, as well as summary statistics and histograms (for quantitative variables), and relative frequency tables and charts (for categorical variables).

You should construct your Data Appendix \underline{as soon as you have finished} writing the \textbf{codes} (in the Codes \(\rightarrow\) Pre-Analysis folder) that create your analysis data files.
When constructing the appendix, you are likely to learn things about your data that you should know before you begin your analysis. That is, you should not begin the analysis until after you have constructed your Data Appendix.\footnote{Andy Rose (Notes for Efficient Data Organization/Handling, 2017): Check your data sets by graphing and looking for outliers. Look for outliers through descriptive statistics (e.g. look for stuff that lies a few standard errors beyond the mean). Time series plots (of both levels and growth rates) are great for time series. Always provide descriptive statistics right before you do any serious empirical work.}\\

\underline{Data File Name}

Contents of the file.\\

Variable name:

Source:

Values:

Description:

Summary statistics:

Chart:\\


Variable name:

Source:

Values:

Description:

Summary statistics:

Chart:\\


\end{document}

